% Options for packages loaded elsewhere
\PassOptionsToPackage{unicode}{hyperref}
\PassOptionsToPackage{hyphens}{url}
%
\documentclass[
]{article}
\usepackage{amsmath,amssymb}
\usepackage{lmodern}
\usepackage{iftex}
\ifPDFTeX
  \usepackage[T1]{fontenc}
  \usepackage[utf8]{inputenc}
  \usepackage{textcomp} % provide euro and other symbols
\else % if luatex or xetex
  \usepackage{unicode-math}
  \defaultfontfeatures{Scale=MatchLowercase}
  \defaultfontfeatures[\rmfamily]{Ligatures=TeX,Scale=1}
\fi
% Use upquote if available, for straight quotes in verbatim environments
\IfFileExists{upquote.sty}{\usepackage{upquote}}{}
\IfFileExists{microtype.sty}{% use microtype if available
  \usepackage[]{microtype}
  \UseMicrotypeSet[protrusion]{basicmath} % disable protrusion for tt fonts
}{}
\makeatletter
\@ifundefined{KOMAClassName}{% if non-KOMA class
  \IfFileExists{parskip.sty}{%
    \usepackage{parskip}
  }{% else
    \setlength{\parindent}{0pt}
    \setlength{\parskip}{6pt plus 2pt minus 1pt}}
}{% if KOMA class
  \KOMAoptions{parskip=half}}
\makeatother
\usepackage{xcolor}
\usepackage[margin=1in]{geometry}
\usepackage{color}
\usepackage{fancyvrb}
\newcommand{\VerbBar}{|}
\newcommand{\VERB}{\Verb[commandchars=\\\{\}]}
\DefineVerbatimEnvironment{Highlighting}{Verbatim}{commandchars=\\\{\}}
% Add ',fontsize=\small' for more characters per line
\usepackage{framed}
\definecolor{shadecolor}{RGB}{248,248,248}
\newenvironment{Shaded}{\begin{snugshade}}{\end{snugshade}}
\newcommand{\AlertTok}[1]{\textcolor[rgb]{0.94,0.16,0.16}{#1}}
\newcommand{\AnnotationTok}[1]{\textcolor[rgb]{0.56,0.35,0.01}{\textbf{\textit{#1}}}}
\newcommand{\AttributeTok}[1]{\textcolor[rgb]{0.77,0.63,0.00}{#1}}
\newcommand{\BaseNTok}[1]{\textcolor[rgb]{0.00,0.00,0.81}{#1}}
\newcommand{\BuiltInTok}[1]{#1}
\newcommand{\CharTok}[1]{\textcolor[rgb]{0.31,0.60,0.02}{#1}}
\newcommand{\CommentTok}[1]{\textcolor[rgb]{0.56,0.35,0.01}{\textit{#1}}}
\newcommand{\CommentVarTok}[1]{\textcolor[rgb]{0.56,0.35,0.01}{\textbf{\textit{#1}}}}
\newcommand{\ConstantTok}[1]{\textcolor[rgb]{0.00,0.00,0.00}{#1}}
\newcommand{\ControlFlowTok}[1]{\textcolor[rgb]{0.13,0.29,0.53}{\textbf{#1}}}
\newcommand{\DataTypeTok}[1]{\textcolor[rgb]{0.13,0.29,0.53}{#1}}
\newcommand{\DecValTok}[1]{\textcolor[rgb]{0.00,0.00,0.81}{#1}}
\newcommand{\DocumentationTok}[1]{\textcolor[rgb]{0.56,0.35,0.01}{\textbf{\textit{#1}}}}
\newcommand{\ErrorTok}[1]{\textcolor[rgb]{0.64,0.00,0.00}{\textbf{#1}}}
\newcommand{\ExtensionTok}[1]{#1}
\newcommand{\FloatTok}[1]{\textcolor[rgb]{0.00,0.00,0.81}{#1}}
\newcommand{\FunctionTok}[1]{\textcolor[rgb]{0.00,0.00,0.00}{#1}}
\newcommand{\ImportTok}[1]{#1}
\newcommand{\InformationTok}[1]{\textcolor[rgb]{0.56,0.35,0.01}{\textbf{\textit{#1}}}}
\newcommand{\KeywordTok}[1]{\textcolor[rgb]{0.13,0.29,0.53}{\textbf{#1}}}
\newcommand{\NormalTok}[1]{#1}
\newcommand{\OperatorTok}[1]{\textcolor[rgb]{0.81,0.36,0.00}{\textbf{#1}}}
\newcommand{\OtherTok}[1]{\textcolor[rgb]{0.56,0.35,0.01}{#1}}
\newcommand{\PreprocessorTok}[1]{\textcolor[rgb]{0.56,0.35,0.01}{\textit{#1}}}
\newcommand{\RegionMarkerTok}[1]{#1}
\newcommand{\SpecialCharTok}[1]{\textcolor[rgb]{0.00,0.00,0.00}{#1}}
\newcommand{\SpecialStringTok}[1]{\textcolor[rgb]{0.31,0.60,0.02}{#1}}
\newcommand{\StringTok}[1]{\textcolor[rgb]{0.31,0.60,0.02}{#1}}
\newcommand{\VariableTok}[1]{\textcolor[rgb]{0.00,0.00,0.00}{#1}}
\newcommand{\VerbatimStringTok}[1]{\textcolor[rgb]{0.31,0.60,0.02}{#1}}
\newcommand{\WarningTok}[1]{\textcolor[rgb]{0.56,0.35,0.01}{\textbf{\textit{#1}}}}
\usepackage{graphicx}
\makeatletter
\def\maxwidth{\ifdim\Gin@nat@width>\linewidth\linewidth\else\Gin@nat@width\fi}
\def\maxheight{\ifdim\Gin@nat@height>\textheight\textheight\else\Gin@nat@height\fi}
\makeatother
% Scale images if necessary, so that they will not overflow the page
% margins by default, and it is still possible to overwrite the defaults
% using explicit options in \includegraphics[width, height, ...]{}
\setkeys{Gin}{width=\maxwidth,height=\maxheight,keepaspectratio}
% Set default figure placement to htbp
\makeatletter
\def\fps@figure{htbp}
\makeatother
\setlength{\emergencystretch}{3em} % prevent overfull lines
\providecommand{\tightlist}{%
  \setlength{\itemsep}{0pt}\setlength{\parskip}{0pt}}
\setcounter{secnumdepth}{-\maxdimen} % remove section numbering
\ifLuaTeX
  \usepackage{selnolig}  % disable illegal ligatures
\fi
\IfFileExists{bookmark.sty}{\usepackage{bookmark}}{\usepackage{hyperref}}
\IfFileExists{xurl.sty}{\usepackage{xurl}}{} % add URL line breaks if available
\urlstyle{same} % disable monospaced font for URLs
\hypersetup{
  pdftitle={Homework3},
  pdfauthor={Yongye Tan},
  hidelinks,
  pdfcreator={LaTeX via pandoc}}

\title{Homework3}
\author{Yongye Tan}
\date{10/22/2022}

\begin{document}
\maketitle

\begin{Shaded}
\begin{Highlighting}[]
\NormalTok{rosca }\OtherTok{\textless{}{-}} \FunctionTok{read.csv}\NormalTok{(}\StringTok{"rosca.csv"}\NormalTok{)}
\end{Highlighting}
\end{Shaded}

\begin{Shaded}
\begin{Highlighting}[]
\NormalTok{rosca}\SpecialCharTok{$}\NormalTok{treatment }\OtherTok{\textless{}{-}} \ConstantTok{NA}
\NormalTok{rosca}\SpecialCharTok{$}\NormalTok{treatment[rosca}\SpecialCharTok{$}\NormalTok{encouragement }\SpecialCharTok{==} \DecValTok{1}\NormalTok{] }\OtherTok{\textless{}{-}} \StringTok{"control"}
\NormalTok{rosca}\SpecialCharTok{$}\NormalTok{treatment[rosca}\SpecialCharTok{$}\NormalTok{safe\_box }\SpecialCharTok{==} \DecValTok{1}\NormalTok{] }\OtherTok{\textless{}{-}} \StringTok{"safebox"}
\NormalTok{rosca}\SpecialCharTok{$}\NormalTok{treatment[rosca}\SpecialCharTok{$}\NormalTok{locked\_box }\SpecialCharTok{==} \DecValTok{1}\NormalTok{] }\OtherTok{\textless{}{-}} \StringTok{"lockbox"}
\NormalTok{rosca}\SpecialCharTok{$}\NormalTok{treatment }\OtherTok{\textless{}{-}} \FunctionTok{as.factor}\NormalTok{(rosca}\SpecialCharTok{$}\NormalTok{treatment)}
\NormalTok{rosca }\OtherTok{\textless{}{-}}\NormalTok{  rosca[rosca}\SpecialCharTok{$}\NormalTok{has\_followup2 }\SpecialCharTok{==} \DecValTok{1}\NormalTok{, ]}
\NormalTok{rosca}\SpecialCharTok{$}\NormalTok{bg\_female }\OtherTok{\textless{}{-}}  \FunctionTok{factor}\NormalTok{(rosca}\SpecialCharTok{$}\NormalTok{bg\_female )}
\NormalTok{rosca}\SpecialCharTok{$}\NormalTok{bg\_married }\OtherTok{\textless{}{-}}  \FunctionTok{factor}\NormalTok{(rosca}\SpecialCharTok{$}\NormalTok{bg\_married)}
\end{Highlighting}
\end{Shaded}

\hypertarget{section}{%
\section{1}\label{section}}

\begin{enumerate}
\def\labelenumi{\alph{enumi}.}
\tightlist
\item
  Use a bar chart to show the distribution of bg\_married within each
  treatment condition. Which treatment condition has the highest
  proportion of married people? (Note: you need to add appropriate
  title, labels of axes, lengend name, and change the names of the two
  levels in the legend.)
\end{enumerate}

The lockbox treament condition has the highest proportion of married
people, as shown with the blue color.

\begin{Shaded}
\begin{Highlighting}[]
\FunctionTok{ggplot}\NormalTok{(}\AttributeTok{data =}\NormalTok{ rosca) }\SpecialCharTok{+} 
  \FunctionTok{geom\_bar}\NormalTok{(}\AttributeTok{mapping =} \FunctionTok{aes}\NormalTok{(treatment, }\AttributeTok{fill =}\NormalTok{ bg\_female), }
           \AttributeTok{position =} \StringTok{"dodge"}\NormalTok{) }\SpecialCharTok{+} 
  \FunctionTok{theme}\NormalTok{(}\AttributeTok{axis.title.x =} \FunctionTok{element\_blank}\NormalTok{()) }\SpecialCharTok{+} 
  \FunctionTok{theme}\NormalTok{(}\AttributeTok{axis.title.y =} \FunctionTok{element\_blank}\NormalTok{()) }\SpecialCharTok{+} 
  \CommentTok{\# add a title }
  \FunctionTok{labs}\NormalTok{(}
    \AttributeTok{title =} \StringTok{"Distribution of marital status"}
\NormalTok{  ) }\SpecialCharTok{+} 
  \CommentTok{\# center the title and bold it}
  \FunctionTok{theme}\NormalTok{(}\AttributeTok{plot.title =} \FunctionTok{element\_text}\NormalTok{(}\AttributeTok{hjust =} \FloatTok{0.5}\NormalTok{, }\AttributeTok{face =} \StringTok{"bold"}\NormalTok{)) }\SpecialCharTok{+} 
  \CommentTok{\# change the legend label title and name}
  \FunctionTok{scale\_fill\_discrete}\NormalTok{(}\AttributeTok{name =} \StringTok{"marital status"}\NormalTok{,}
                      \AttributeTok{labels =} \FunctionTok{c}\NormalTok{(}\StringTok{"unmarried"}\NormalTok{, }\StringTok{"married"}\NormalTok{))}
\end{Highlighting}
\end{Shaded}

\includegraphics{Homework3_files/figure-latex/unnamed-chunk-3-1.pdf}

\hypertarget{b.}{%
\subsubsection{b.}\label{b.}}

The barplots do not contain the exact numbers or proportions so we can
add them to make the figure more informative. Make the following plot.
Use geom\_text to add the percentage for each bar (You don't need to be
perfect for the position of the annotation). The following codes
calculate the counts and proportions of unmarried people under control,
married people under control, unmarried people under lockbox, married
people under lockbox, unmarried people under safebox, and married people
under safebox. You need to change the proportions to the percentages.
(You can use other methods for this. Check function percent() in scales
package )

\begin{Shaded}
\begin{Highlighting}[]
\NormalTok{y }\OtherTok{\textless{}{-}} \FunctionTok{as.numeric}\NormalTok{(}\FunctionTok{t}\NormalTok{(}\FunctionTok{table}\NormalTok{(rosca}\SpecialCharTok{$}\NormalTok{treatment,rosca}\SpecialCharTok{$}\NormalTok{bg\_married)))}
\NormalTok{prop }\OtherTok{\textless{}{-}} \FunctionTok{table}\NormalTok{(rosca}\SpecialCharTok{$}\NormalTok{treatment,rosca}\SpecialCharTok{$}\NormalTok{bg\_married)}\SpecialCharTok{/}\FunctionTok{length}\NormalTok{(rosca}\SpecialCharTok{$}\NormalTok{treatment)}
\NormalTok{prop }\OtherTok{\textless{}{-}} \FunctionTok{as.numeric}\NormalTok{(}\FunctionTok{t}\NormalTok{(prop))}
\end{Highlighting}
\end{Shaded}

\begin{Shaded}
\begin{Highlighting}[]
\FunctionTok{ggplot}\NormalTok{(}\AttributeTok{data =}\NormalTok{ rosca) }\SpecialCharTok{+} 
  \FunctionTok{geom\_bar}\NormalTok{(}\AttributeTok{mapping =} \FunctionTok{aes}\NormalTok{(treatment, }\AttributeTok{fill =}\NormalTok{ bg\_female), }
           \AttributeTok{position =} \StringTok{"dodge"}\NormalTok{) }\SpecialCharTok{+} 
  \FunctionTok{theme}\NormalTok{(}\AttributeTok{axis.title.x =} \FunctionTok{element\_blank}\NormalTok{()) }\SpecialCharTok{+} 
  \FunctionTok{theme}\NormalTok{(}\AttributeTok{axis.title.y =} \FunctionTok{element\_blank}\NormalTok{()) }\SpecialCharTok{+} 
  \CommentTok{\# add a title }
  \FunctionTok{labs}\NormalTok{(}
    \AttributeTok{title =} \StringTok{"Distribution of marital status"}
\NormalTok{  ) }\SpecialCharTok{+} 
  \CommentTok{\# center the title and bold it}
  \FunctionTok{theme}\NormalTok{(}\AttributeTok{plot.title =} \FunctionTok{element\_text}\NormalTok{(}\AttributeTok{hjust =} \FloatTok{0.5}\NormalTok{, }\AttributeTok{face =} \StringTok{"bold"}\NormalTok{)) }\SpecialCharTok{+} 
  \CommentTok{\# change the legend label title and name}
  \FunctionTok{scale\_fill\_discrete}\NormalTok{(}\AttributeTok{name =} \StringTok{"marital status"}\NormalTok{,}
                      \AttributeTok{labels =} \FunctionTok{c}\NormalTok{(}\StringTok{"unmarried"}\NormalTok{, }\StringTok{"married"}\NormalTok{)) }
\end{Highlighting}
\end{Shaded}

\includegraphics{Homework3_files/figure-latex/unnamed-chunk-5-1.pdf}

\#2

\begin{Shaded}
\begin{Highlighting}[]
\CommentTok{\# filter out the temp that is na}
\NormalTok{weather }\OtherTok{\textless{}{-}}\NormalTok{ weather }\SpecialCharTok{\%\textgreater{}\%} \FunctionTok{filter}\NormalTok{(}\SpecialCharTok{!}\FunctionTok{is.na}\NormalTok{(temp))}
\NormalTok{temp.facet }\OtherTok{\textless{}{-}} \FunctionTok{ggplot}\NormalTok{(}\AttributeTok{data =}\NormalTok{ weather)}\SpecialCharTok{+}
  \FunctionTok{geom\_histogram}\NormalTok{(}\AttributeTok{mapping =} \FunctionTok{aes}\NormalTok{(}\AttributeTok{x =}\NormalTok{ temp),}\AttributeTok{color=} \StringTok{"white"}\NormalTok{,}\AttributeTok{binwidth =} \DecValTok{5}\NormalTok{)}\SpecialCharTok{+}
  \CommentTok{\# total 4 rows}
  \FunctionTok{facet\_wrap}\NormalTok{(}\SpecialCharTok{\textasciitilde{}}\NormalTok{month,}\AttributeTok{nrow=}\DecValTok{4}\NormalTok{)}
\NormalTok{temp.facet}
\end{Highlighting}
\end{Shaded}

\includegraphics{Homework3_files/figure-latex/unnamed-chunk-6-1.pdf}

\begin{enumerate}
\def\labelenumi{\alph{enumi}.}
\tightlist
\item
  Add appropriate title and labels of axes. Remove the y-axis ticks and
  tick labels (200,400\ldots)
\end{enumerate}

\begin{Shaded}
\begin{Highlighting}[]
\NormalTok{temp.facet}\SpecialCharTok{+} \FunctionTok{labs}\NormalTok{(}
  \AttributeTok{title =} \StringTok{"Temperature in 12 months"}\NormalTok{,}
  \AttributeTok{x =} \StringTok{"Temperature (in Fahrenheit)"}\NormalTok{,}
  \CommentTok{\# remove the y{-}axis ticks}
  \AttributeTok{y =} \ConstantTok{NULL}\NormalTok{) }\SpecialCharTok{+} 
  \CommentTok{\# remove the y tick labels}
  \FunctionTok{scale\_y\_continuous}\NormalTok{(}\AttributeTok{labels =} \ConstantTok{NULL}\NormalTok{)}
\end{Highlighting}
\end{Shaded}

\includegraphics{Homework3_files/figure-latex/unnamed-chunk-7-1.pdf}

\begin{enumerate}
\def\labelenumi{\alph{enumi}.}
\setcounter{enumi}{1}
\tightlist
\item
  Change the facet labels from numeric month to month abbreviations,
  e.g., 1 to Jan.
\end{enumerate}

\begin{Shaded}
\begin{Highlighting}[]
\CommentTok{\# c(1,2,3,4,5,6,7,8,9,10,11,12)}
\NormalTok{month\_label }\OtherTok{\textless{}{-}} \FunctionTok{c}\NormalTok{ (}
                \StringTok{\textquotesingle{}1\textquotesingle{}} \OtherTok{=} \StringTok{"Jan"}\NormalTok{, }\StringTok{\textquotesingle{}2\textquotesingle{}} \OtherTok{=} \StringTok{"Feb"}\NormalTok{,}\StringTok{\textquotesingle{}3\textquotesingle{}} \OtherTok{=} \StringTok{"Mar"}\NormalTok{,}\StringTok{\textquotesingle{}4\textquotesingle{}} \OtherTok{=} \StringTok{"Apr"}\NormalTok{,}
                \StringTok{\textquotesingle{}5\textquotesingle{}} \OtherTok{=} \StringTok{"May"}\NormalTok{, }\StringTok{\textquotesingle{}6\textquotesingle{}} \OtherTok{=} \StringTok{"Jun"}\NormalTok{,}\StringTok{\textquotesingle{}7\textquotesingle{}} \OtherTok{=} \StringTok{"Jul"}\NormalTok{,}\StringTok{\textquotesingle{}8\textquotesingle{}} \OtherTok{=} \StringTok{"Aug"}\NormalTok{,}
                \StringTok{\textquotesingle{}9\textquotesingle{}} \OtherTok{=} \StringTok{"Sep"}\NormalTok{, }\StringTok{\textquotesingle{}10\textquotesingle{}} \OtherTok{=} \StringTok{"Oct"}\NormalTok{,}\StringTok{\textquotesingle{}11\textquotesingle{}} \OtherTok{=} \StringTok{"Nov"}\NormalTok{,}\StringTok{\textquotesingle{}12\textquotesingle{}} \OtherTok{=} \StringTok{"Dec"}\NormalTok{)}


\NormalTok{month\_label\_plot }\OtherTok{\textless{}{-}} \FunctionTok{ggplot}\NormalTok{(}\AttributeTok{data =}\NormalTok{ weather)}\SpecialCharTok{+}
  \FunctionTok{geom\_histogram}\NormalTok{(}\AttributeTok{mapping =} \FunctionTok{aes}\NormalTok{(}\AttributeTok{x =}\NormalTok{ temp),}\AttributeTok{color=} \StringTok{"white"}\NormalTok{,}\AttributeTok{binwidth =} \DecValTok{5}\NormalTok{) }\SpecialCharTok{+}
  \CommentTok{\# total 4 rows}
  \FunctionTok{facet\_wrap}\NormalTok{(}\SpecialCharTok{\textasciitilde{}}\NormalTok{month,}
             \AttributeTok{nrow=}\DecValTok{4}\NormalTok{,}
             \CommentTok{\# match the label to the new label}
             \AttributeTok{labeller =} \FunctionTok{as\_labeller}\NormalTok{(month\_label))}\SpecialCharTok{+} 
  \FunctionTok{labs}\NormalTok{(}
      \AttributeTok{title =} \StringTok{"Temperature in 12 months"}\NormalTok{,}
      \AttributeTok{x =} \StringTok{"Temperature (in Fahrenheit)"}\NormalTok{,}
      \AttributeTok{y =} \ConstantTok{NULL}\NormalTok{) }\SpecialCharTok{+} 
      \FunctionTok{scale\_y\_continuous}\NormalTok{(}\AttributeTok{labels =} \ConstantTok{NULL}\NormalTok{) }
\NormalTok{month\_label\_plot}
\end{Highlighting}
\end{Shaded}

\includegraphics{Homework3_files/figure-latex/unnamed-chunk-8-1.pdf}

\begin{enumerate}
\def\labelenumi{\alph{enumi}.}
\setcounter{enumi}{2}
\tightlist
\item
  We can add the average termperature for each month to make the firgure
  more informative. Make the following plot. You can use
  round(tapply(weather\(temp, factor(weather\)month),mean)) to obtain
  the average temperature of each month.
\end{enumerate}

\begin{Shaded}
\begin{Highlighting}[]
\NormalTok{month\_label }\OtherTok{\textless{}{-}} \FunctionTok{c}\NormalTok{ (}
                \StringTok{\textquotesingle{}1\textquotesingle{}} \OtherTok{=} \StringTok{"Jan"}\NormalTok{, }\StringTok{\textquotesingle{}2\textquotesingle{}} \OtherTok{=} \StringTok{"Feb"}\NormalTok{,}\StringTok{\textquotesingle{}3\textquotesingle{}} \OtherTok{=} \StringTok{"Mar"}\NormalTok{,}\StringTok{\textquotesingle{}4\textquotesingle{}} \OtherTok{=} \StringTok{"Apr"}\NormalTok{,}
                \StringTok{\textquotesingle{}5\textquotesingle{}} \OtherTok{=} \StringTok{"May"}\NormalTok{, }\StringTok{\textquotesingle{}6\textquotesingle{}} \OtherTok{=} \StringTok{"Jun"}\NormalTok{,}\StringTok{\textquotesingle{}7\textquotesingle{}} \OtherTok{=} \StringTok{"Jul"}\NormalTok{,}\StringTok{\textquotesingle{}8\textquotesingle{}} \OtherTok{=} \StringTok{"Aug"}\NormalTok{,}
                \StringTok{\textquotesingle{}9\textquotesingle{}} \OtherTok{=} \StringTok{"Sep"}\NormalTok{, }\StringTok{\textquotesingle{}10\textquotesingle{}} \OtherTok{=} \StringTok{"Oct"}\NormalTok{,}\StringTok{\textquotesingle{}11\textquotesingle{}} \OtherTok{=} \StringTok{"Nov"}\NormalTok{,}\StringTok{\textquotesingle{}12\textquotesingle{}} \OtherTok{=} \StringTok{"Dec"}\NormalTok{)}

\NormalTok{temperature\_mean }\OtherTok{\textless{}{-}} \FunctionTok{round}\NormalTok{(}\FunctionTok{tapply}\NormalTok{(weather}\SpecialCharTok{$}\NormalTok{temp, }\FunctionTok{factor}\NormalTok{(weather}\SpecialCharTok{$}\NormalTok{month),mean))}
\NormalTok{temp\_text }\OtherTok{\textless{}{-}} \FunctionTok{data.frame}\NormalTok{(}
  \CommentTok{\# this is important (help the label to find where it belongs 1 {-}\textgreater{} mean 34, 2 {-}\textgreater{} mean 32)}
  \AttributeTok{month =} \FunctionTok{c}\NormalTok{(}\DecValTok{1}\NormalTok{,}\DecValTok{2}\NormalTok{,}\DecValTok{3}\NormalTok{,}\DecValTok{4}\NormalTok{,}\DecValTok{5}\NormalTok{,}\DecValTok{6}\NormalTok{,}\DecValTok{7}\NormalTok{,}\DecValTok{8}\NormalTok{,}\DecValTok{9}\NormalTok{,}\DecValTok{10}\NormalTok{,}\DecValTok{11}\NormalTok{,}\DecValTok{12}\NormalTok{),}
  \CommentTok{\# set the text location}
  \AttributeTok{x =} \DecValTok{90}\NormalTok{,}
  \AttributeTok{y =} \DecValTok{200}\NormalTok{,}
  \AttributeTok{label =} \FunctionTok{str\_c}\NormalTok{(}\StringTok{"mean="}\NormalTok{,temperature\_mean)}
\NormalTok{)}

\NormalTok{graph }\OtherTok{\textless{}{-}} \FunctionTok{ggplot}\NormalTok{(}\AttributeTok{data =}\NormalTok{ weather, }\FunctionTok{aes}\NormalTok{(}\AttributeTok{x =}\NormalTok{ temp )) }\SpecialCharTok{+}
  \FunctionTok{geom\_histogram}\NormalTok{(}\AttributeTok{color=} \StringTok{"white"}\NormalTok{,}\AttributeTok{binwidth =} \DecValTok{5}\NormalTok{) }\SpecialCharTok{+}
  \CommentTok{\# total 4 rows}
  \FunctionTok{facet\_wrap}\NormalTok{(}\SpecialCharTok{\textasciitilde{}}\NormalTok{month,}
             \AttributeTok{nrow=}\DecValTok{4}\NormalTok{,}
             \CommentTok{\# match the label to the new label}
             \AttributeTok{labeller =} \FunctionTok{as\_labeller}\NormalTok{(month\_label)}
\NormalTok{             ) }\SpecialCharTok{+} 
  \FunctionTok{labs}\NormalTok{(}
      \AttributeTok{title =} \StringTok{"Temperature in 12 months"}\NormalTok{,}
      \AttributeTok{x =} \StringTok{"Temperature (in Fahrenheit)"}\NormalTok{,}
      \AttributeTok{y =} \ConstantTok{NULL}\NormalTok{) }\SpecialCharTok{+} 
      \FunctionTok{scale\_y\_continuous}\NormalTok{(}\AttributeTok{labels =} \ConstantTok{NULL}\NormalTok{) }

\NormalTok{graph }\SpecialCharTok{+}
     \FunctionTok{geom\_text}\NormalTok{(}
       \AttributeTok{data =}\NormalTok{ temp\_text,}
       \AttributeTok{mapping =} \FunctionTok{aes}\NormalTok{(}\AttributeTok{x =}\NormalTok{ x, }\AttributeTok{y =}\NormalTok{ y,}
                     \AttributeTok{label =}\NormalTok{ label)}
\NormalTok{     )}
\end{Highlighting}
\end{Shaded}

\includegraphics{Homework3_files/figure-latex/unnamed-chunk-9-1.pdf}

\hypertarget{section-1}{%
\section{3}\label{section-1}}

\begin{Shaded}
\begin{Highlighting}[]
\FunctionTok{ggplot}\NormalTok{(}\AttributeTok{data =}\NormalTok{ weather, }\AttributeTok{mapping =} \FunctionTok{aes}\NormalTok{(}\AttributeTok{x=} \FunctionTok{factor}\NormalTok{(month),}\AttributeTok{y =}\NormalTok{ temp)) }\SpecialCharTok{+}
  \FunctionTok{geom\_boxplot}\NormalTok{()}
\end{Highlighting}
\end{Shaded}

\includegraphics{Homework3_files/figure-latex/unnamed-chunk-10-1.pdf}

\begin{enumerate}
\def\labelenumi{\alph{enumi}.}
\tightlist
\item
  Remove the x-axis label, and change the tick labels from numeric month
  to month abbreviation as in the following plot.
\end{enumerate}

\begin{Shaded}
\begin{Highlighting}[]
\NormalTok{month\_label }\OtherTok{\textless{}{-}} \FunctionTok{c}\NormalTok{ (}
                \StringTok{\textquotesingle{}1\textquotesingle{}} \OtherTok{=} \StringTok{"Jan"}\NormalTok{, }\StringTok{\textquotesingle{}2\textquotesingle{}} \OtherTok{=} \StringTok{"Feb"}\NormalTok{,}\StringTok{\textquotesingle{}3\textquotesingle{}} \OtherTok{=} \StringTok{"Mar"}\NormalTok{,}\StringTok{\textquotesingle{}4\textquotesingle{}} \OtherTok{=} \StringTok{"Apr"}\NormalTok{,}
                \StringTok{\textquotesingle{}5\textquotesingle{}} \OtherTok{=} \StringTok{"May"}\NormalTok{, }\StringTok{\textquotesingle{}6\textquotesingle{}} \OtherTok{=} \StringTok{"Jun"}\NormalTok{,}\StringTok{\textquotesingle{}7\textquotesingle{}} \OtherTok{=} \StringTok{"Jul"}\NormalTok{,}\StringTok{\textquotesingle{}8\textquotesingle{}} \OtherTok{=} \StringTok{"Aug"}\NormalTok{,}
                \StringTok{\textquotesingle{}9\textquotesingle{}} \OtherTok{=} \StringTok{"Sep"}\NormalTok{, }\StringTok{\textquotesingle{}10\textquotesingle{}} \OtherTok{=} \StringTok{"Oct"}\NormalTok{,}\StringTok{\textquotesingle{}11\textquotesingle{}} \OtherTok{=} \StringTok{"Nov"}\NormalTok{,}\StringTok{\textquotesingle{}12\textquotesingle{}} \OtherTok{=} \StringTok{"Dec"}\NormalTok{)}
\FunctionTok{ggplot}\NormalTok{(}\AttributeTok{data =}\NormalTok{ weather, }\AttributeTok{mapping =} \FunctionTok{aes}\NormalTok{(}\AttributeTok{x=} \FunctionTok{factor}\NormalTok{(month),}\AttributeTok{y =}\NormalTok{ temp)) }\SpecialCharTok{+}
  \FunctionTok{geom\_boxplot}\NormalTok{() }\SpecialCharTok{+} 
  \FunctionTok{labs}\NormalTok{(}\AttributeTok{x=} \ConstantTok{NULL}\NormalTok{) }\SpecialCharTok{+} 
  \CommentTok{\# since these month are discrete}
  \FunctionTok{scale\_x\_discrete}\NormalTok{(}\AttributeTok{label =}\NormalTok{ month\_label) }
\end{Highlighting}
\end{Shaded}

\includegraphics{Homework3_files/figure-latex/unnamed-chunk-11-1.pdf}

\begin{enumerate}
\def\labelenumi{\alph{enumi}.}
\setcounter{enumi}{1}
\tightlist
\item
  Remove the y-axis label, and change the tick labels on the y-axis as
  in the following plot.
\end{enumerate}

\begin{Shaded}
\begin{Highlighting}[]
\NormalTok{month\_label }\OtherTok{\textless{}{-}} \FunctionTok{c}\NormalTok{ (}
                \StringTok{\textquotesingle{}1\textquotesingle{}} \OtherTok{=} \StringTok{"Jan"}\NormalTok{, }\StringTok{\textquotesingle{}2\textquotesingle{}} \OtherTok{=} \StringTok{"Feb"}\NormalTok{,}\StringTok{\textquotesingle{}3\textquotesingle{}} \OtherTok{=} \StringTok{"Mar"}\NormalTok{,}\StringTok{\textquotesingle{}4\textquotesingle{}} \OtherTok{=} \StringTok{"Apr"}\NormalTok{,}
                \StringTok{\textquotesingle{}5\textquotesingle{}} \OtherTok{=} \StringTok{"May"}\NormalTok{, }\StringTok{\textquotesingle{}6\textquotesingle{}} \OtherTok{=} \StringTok{"Jun"}\NormalTok{,}\StringTok{\textquotesingle{}7\textquotesingle{}} \OtherTok{=} \StringTok{"Jul"}\NormalTok{,}\StringTok{\textquotesingle{}8\textquotesingle{}} \OtherTok{=} \StringTok{"Aug"}\NormalTok{,}
                \StringTok{\textquotesingle{}9\textquotesingle{}} \OtherTok{=} \StringTok{"Sep"}\NormalTok{, }\StringTok{\textquotesingle{}10\textquotesingle{}} \OtherTok{=} \StringTok{"Oct"}\NormalTok{,}\StringTok{\textquotesingle{}11\textquotesingle{}} \OtherTok{=} \StringTok{"Nov"}\NormalTok{,}\StringTok{\textquotesingle{}12\textquotesingle{}} \OtherTok{=} \StringTok{"Dec"}\NormalTok{)}

\FunctionTok{ggplot}\NormalTok{(}\AttributeTok{data =}\NormalTok{ weather, }\AttributeTok{mapping =} \FunctionTok{aes}\NormalTok{(}\AttributeTok{x=} \FunctionTok{factor}\NormalTok{(month),}\AttributeTok{y =}\NormalTok{ temp)) }\SpecialCharTok{+}
  \FunctionTok{geom\_boxplot}\NormalTok{() }\SpecialCharTok{+} 
  \FunctionTok{labs}\NormalTok{(}\AttributeTok{x=} \ConstantTok{NULL}\NormalTok{,}
       \AttributeTok{y =} \ConstantTok{NULL}\NormalTok{) }\SpecialCharTok{+} 
  \CommentTok{\# since these months are discrete}
  \FunctionTok{scale\_x\_discrete}\NormalTok{(}\AttributeTok{label =}\NormalTok{ month\_label) }\SpecialCharTok{+}
  \CommentTok{\# have to set the y axis mannually and and change the label mannually}
  \FunctionTok{scale\_y\_continuous}\NormalTok{(}\AttributeTok{breaks =} \FunctionTok{seq}\NormalTok{(}\DecValTok{0}\NormalTok{, }\FunctionTok{max}\NormalTok{(weather}\SpecialCharTok{$}\NormalTok{temp, }\AttributeTok{na.rm=}\ConstantTok{TRUE}\NormalTok{), }\AttributeTok{by =} \DecValTok{25}\NormalTok{),}
                     \AttributeTok{label =} \FunctionTok{c}\NormalTok{(}\StringTok{"0"} \OtherTok{=} \StringTok{"0F"}\NormalTok{,}
                               \StringTok{"25"} \OtherTok{=} \StringTok{"25F"}\NormalTok{,}
                               \StringTok{"50"} \OtherTok{=} \StringTok{"50F"}\NormalTok{,}
                               \StringTok{"75"} \OtherTok{=} \StringTok{"75F"}\NormalTok{,}
                               \StringTok{"100"} \OtherTok{=} \StringTok{"100F"}\NormalTok{))}
\end{Highlighting}
\end{Shaded}

\includegraphics{Homework3_files/figure-latex/unnamed-chunk-12-1.pdf}

\begin{enumerate}
\def\labelenumi{\alph{enumi}.}
\setcounter{enumi}{2}
\tightlist
\item
  (optional) We can make it better by adding the degree symbol. Make the
  following plot.
\end{enumerate}

\begin{Shaded}
\begin{Highlighting}[]
\NormalTok{month\_label }\OtherTok{\textless{}{-}} \FunctionTok{c}\NormalTok{ (}
                \StringTok{\textquotesingle{}1\textquotesingle{}} \OtherTok{=} \StringTok{"Jan"}\NormalTok{, }\StringTok{\textquotesingle{}2\textquotesingle{}} \OtherTok{=} \StringTok{"Feb"}\NormalTok{,}\StringTok{\textquotesingle{}3\textquotesingle{}} \OtherTok{=} \StringTok{"Mar"}\NormalTok{,}\StringTok{\textquotesingle{}4\textquotesingle{}} \OtherTok{=} \StringTok{"Apr"}\NormalTok{,}
                \StringTok{\textquotesingle{}5\textquotesingle{}} \OtherTok{=} \StringTok{"May"}\NormalTok{, }\StringTok{\textquotesingle{}6\textquotesingle{}} \OtherTok{=} \StringTok{"Jun"}\NormalTok{,}\StringTok{\textquotesingle{}7\textquotesingle{}} \OtherTok{=} \StringTok{"Jul"}\NormalTok{,}\StringTok{\textquotesingle{}8\textquotesingle{}} \OtherTok{=} \StringTok{"Aug"}\NormalTok{,}
                \StringTok{\textquotesingle{}9\textquotesingle{}} \OtherTok{=} \StringTok{"Sep"}\NormalTok{, }\StringTok{\textquotesingle{}10\textquotesingle{}} \OtherTok{=} \StringTok{"Oct"}\NormalTok{,}\StringTok{\textquotesingle{}11\textquotesingle{}} \OtherTok{=} \StringTok{"Nov"}\NormalTok{,}\StringTok{\textquotesingle{}12\textquotesingle{}} \OtherTok{=} \StringTok{"Dec"}\NormalTok{)}

\FunctionTok{ggplot}\NormalTok{(}\AttributeTok{data =}\NormalTok{ weather, }\AttributeTok{mapping =} \FunctionTok{aes}\NormalTok{(}\AttributeTok{x=} \FunctionTok{factor}\NormalTok{(month),}\AttributeTok{y =}\NormalTok{ temp)) }\SpecialCharTok{+}
  \FunctionTok{geom\_boxplot}\NormalTok{() }\SpecialCharTok{+} 
  \FunctionTok{labs}\NormalTok{(}\AttributeTok{x=} \ConstantTok{NULL}\NormalTok{,}
       \AttributeTok{y =} \ConstantTok{NULL}\NormalTok{) }\SpecialCharTok{+} 
  \CommentTok{\# since these months are discrete}
  \FunctionTok{scale\_x\_discrete}\NormalTok{(}\AttributeTok{label =}\NormalTok{ month\_label) }\SpecialCharTok{+}
  \CommentTok{\# have to set the y axis mannually and and change the label mannually}
  \FunctionTok{scale\_y\_continuous}\NormalTok{(}\AttributeTok{breaks =} \FunctionTok{seq}\NormalTok{(}\DecValTok{0}\NormalTok{, }\FunctionTok{max}\NormalTok{(weather}\SpecialCharTok{$}\NormalTok{temp, }\AttributeTok{na.rm=}\ConstantTok{TRUE}\NormalTok{), }\AttributeTok{by =} \DecValTok{25}\NormalTok{),}
                     \AttributeTok{label =} \FunctionTok{c}\NormalTok{(}\StringTok{"0"} \OtherTok{=} \StringTok{"0°F"}\NormalTok{,}
                               \StringTok{"25"} \OtherTok{=} \StringTok{"25°F"}\NormalTok{,}
                               \StringTok{"50"} \OtherTok{=} \StringTok{"50°F"}\NormalTok{,}
                               \StringTok{"75"} \OtherTok{=} \StringTok{"75°F"}\NormalTok{,}
                               \StringTok{"100"} \OtherTok{=} \StringTok{"100°F"}\NormalTok{))}
\end{Highlighting}
\end{Shaded}

\includegraphics{Homework3_files/figure-latex/unnamed-chunk-13-1.pdf}

\end{document}
